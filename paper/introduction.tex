Natural sciences are tasked to find, from a set of hypothetical factors, the minimal subset that suffices to reliably predict novel observations. This endeavor is impeded by two major challenges.

First, causal and non-causal factors may be numerous and collinear. In physics, for example, one may be challenged to identify whether fusion is caused by a change in temperature or a change in pressure, as these two factors may, at first, be difficult to manipulate independently. This issue becomes increasingly pronounced as the number of potential factors increases. In neuroscience, for example, identifying whether the frequency of a word presented on a subject's retina modulates brain activity can be surprisingly difficult. Indeed, the frequency of words in natural language covaries with other factors such as their length (short words are more frequent than long words), their categories (determinants are more frequent than adverbs) and so forth \citep{kutas2011thirty,pegado2014timing}. Instead of selecting a set of words that control for all of these factors simultaneously, it is thus common to use forward modeling, i.e. to train a model to predict observations (e.g. brain activity) from a minimal combination of competing factors (e.g. word length, word frequency, e.g. \citep{huth2016natural}), and investigate, in the model, the estimated contribution of each factor \citep{friston1994statistical}.

The second challenge to measuring causal influence is that observations can be large and complex. The relationship between causes and effects is thus often considered in a backward manner, by training models to maximally predict causes from multidimensional observations. For example, brain activity is often recorded with hundreds or thousands of sensors simultaneously. As multiple sensors may be affected by common noise sources, it is common to use backward regression, by, for example, fitting a support vector machine across multiple sensors to decode the category of a stimulus \citep{cichy2014resolving,  kriegeskorte2008representational, norman2006beyond}.

Both forward and backward modeling have competing benefits and drawbacks. Specifically, forward modeling disentangles the independent contribution of collinear factors, but does not combine multidimensional observations. By contrast, backward modeling combines multiple observations, but does not disentangle collinear factors \citep{weichwald2015causal, hebart2018deconstructing, king2018encoding}. To combine the benefits of forward and backward modeling, several authors have proposed to use cross-decomposition techniques such as Partial Least Squares (PLS) and Canonical Correlation Analysis (CCA) \citep{de2019multiway}.
CCA and PLS aim to find, from two sets of data $X$ and $Y$, the components $H$ and $G$ were $XH$ and $YG$ are maximially correlated or maximally covarying respectively. Because CCA and PLS are based on a generalized eigen decomposition, their resulting coefficients are mixing the features of $X$ and $Y$ in a way that makes them notoriously difficult to interpret \citep{lebart1995statistique}.

Here, we introduce the `back-to-back regression' (B2B), which not only combines the benefits of forward and backward modeling (Section~\ref{sec:algorithm}), but also provide robust and interpretable causal coefficients. After detailing the method and proving its convergence (Section~\ref{sec:theorem}), we show with synthetic data that it outperforms state-of-the-art forward, backward and cross-decomposition techniques in identifying causal influence (Section~\ref{sec:experiment_synthetic}). Finally, we apply B2B to a large neuroimaging dataset and reveal that distinct but collinear word features lead to distinguishable brain representations (Section~\ref{sec:experiment_real}).
