\documentclass{article}

\usepackage{amsmath, amsfonts, microtype, xcolor, tikz, graphicx, hyperref, amsthm}
\usepackage[ruled, linesnumbered]{algorithm2e}
\usepackage[]{neurips_2019}

\title{Measuring causal influence with\\ back-to-back regression: the linear case - supplementary material}

\begin{document}

\maketitle

\begin{figure}[h]
  \centering
  \includegraphics[width=\textwidth, trim=0cm 0cm 0cm 0cm, clip=True]{figures/meg_sensors.pdf}
  \caption{Magnetosensor response (femtoteslas fT) averaged across words for the first subject. The color coding corresponds to the positions of the sensors on the head as shown on the top-left diagram. The diverging curves correspond to artifacts in the MEG data.}
  \label{fig:megavg}
\end{figure}


\begin{figure}[h]
  \centering
  \includegraphics[width=\textwidth, trim=0cm 0cm 0cm 0cm, clip=True]{figures/ridgecv_baseline_result.pdf}
  \caption{Regression baseline. Coefficients appear to be
  linearly correlated, unlike our method in the main text.}
  \label{fig:ridgebaselineresult}
\end{figure}


\begin{figure}[h]
  \centering
  \includegraphics[width=\textwidth, trim=0cm 0cm 0cm 0cm]{figures/pvalues_vertical.pdf}
  \caption{We show the average value and the standard deviation of the coefficients obtained with out method during the trials. the $\star$  symbol corresponds to statistically significant (t-test) nonzero values. We notice in particular that the letters q, x, y are under-represented, as expected. Similarly there were only few proper nouns (PROPN) in text corpora. }
  \label{fig:ridgebaselineresult}
\end{figure}


\end{document}
